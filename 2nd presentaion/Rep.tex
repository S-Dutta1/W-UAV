\documentclass{beamer}
\usepackage{color}
\usepackage{booktabs}
\usepackage{multirow}
\usepackage{amssymb,graphicx}
\usepackage{pifont}

\begin{document}

\title{Multi-UAV Simulation Presentation}   
\author{Srijit Dutta\\} 
\date{\today} 


\frame{\titlepage} 
\section{Report} 



\subsection{Box2D}
\frame{\frametitle{Improvement from previous simulation} 
\begin{itemize}
\item A 2-D model has been developed, this is based on the assumption that the height is constant for all UAVs
\item All UAVs keep monitoring all fires upto a time the 'false alarm' is detected
\item This false alarm is selected randomly and the UAVs adjust their course accordingly
\item Also self collisions among UAVs have been neglected
\item Obstale avoiding has been implented in some cases as fires have been set randomly, thus setting a random obstacle in most cases causes overlap
\end{itemize}
}
%%%%%%%%%%%%%%%%%%%
\frame{\frametitle{Continued..} 
\begin{itemize}
\item The path-planning algorithm discussed last time has been modified a little
\item The priority of the fires/cells are based on the "age"/time it has been unattended for 
\item This makes sure no fire is neglected for a long time
\item Also the parameter of distance in the original algorithm has been replaced by moving to the next unattended fire
\item This does not create a problem when path of 2 UAVs cross
\end{itemize}
}

%%%%%%%%%%%%%%%%%%%%%%%%%%
\frame{\frametitle{Simulation} 
\begin{itemize}
\item The UAVs return to their base station after fire monitoring
\end{itemize}

\begin{figure}[h!]
\centering
 \includegraphics[width=80mm]{motion1.png}
 \caption{Monitoring}
 \label{fig:boat1}
\end{figure}
}

%%%%%%%%%%%%%%%%%%%%%%%%%
\frame{\frametitle{Simulation} 
\begin{itemize}
\item Path planning is adjusted once false fire is detected
\end{itemize}

\begin{figure}[h!]
\centering
 \includegraphics[width=80mm]{fixed1.png}
 \caption{New Arrangement}
 \label{fig:boat2}
\end{figure}
}

\end{document}